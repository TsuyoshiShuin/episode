% \documentclass{jsarticle}
\usepackage{amsmath}
\usepackage{amssymb, mathtools, empheq}%the last one is for displaying system for equations
\usepackage{mathrsfs}
\usepackage{amsthm}
\usepackage{bbm}
% \usepackage[math-style=ISO]{unicode-math} --- needs LuaLaTeX or XeTeX
% \usepackage{pxfonts}
\usepackage{comment}%comment enviromment fobiddens tabbing before \begin and \end{comment}
\usepackage{subfiles,docmute}
\usepackage[]{multicol} %multicolum enviroment
% \usepackage[dvips]{graphicx,color} obsolute
\usepackage[dvipdfmx]{graphicx,color}
\usepackage[svgnames,dvipdfmx]{xcolor}
\usepackage{caption}
%\usepackage{pdfpages}
\usepackage{float}
\usepackage{wrapfig}
\usepackage{tikz}
\usepackage{ulem}
\usepackage{udline} % udline is for Japanese
% \usepackage{otf} %roman number mojibake
\usepackage{listings}
\usepackage{jlisting}
\usepackage{bxpapersize}
\usepackage{ascmac} %enclosure
\usepackage[at]{easylist}
\renewcommand{\prepartname}{Part}
\renewcommand{\postpartname}{}
\renewcommand{\thepart}{\arabic{part}}
\renewcommand{\refname}{参考文献} % this can change the section of reference
\usepackage{url}
\usepackage[dvipdfmx]{pdfpages}
\usepackage{chapterbib}
\setcounter{footnote}{0}
\setcounter{section}{0}
\setcounter{figure}{0}
\setcounter{table}{0}
\makeatletter
\newcommand{\subsubsubsection}{\@startsection{paragraph}{4}{\z@}%
  {1.0\Cvs \@plus.5\Cdp \@minus.2\Cdp}%
  {.1\Cvs \@plus.3\Cdp}%
  {\reset@font\sffamily\normalsize}
}
\makeatother
\setcounter{secnumdepth}{4}

% add section numblings to  equation numblings
\makeatletter
\renewcommand{\theequation}{%
  \thesection.\arabic{equation}}
\@addtoreset{equation}{section}
\makeatother


\theoremstyle{definition} %some magic words to make letters in roman in thm enviromment
\newtheorem{thm}{Thm}[section]
\newtheorem{defi}[thm]{Def}
\newtheorem{prop}[thm]{Prop}
\newtheorem{lem}[thm]{Lemma}
\newtheorem{cor}[thm]{Cor}
\newtheorem{rem}[thm]{Rem}
\renewcommand{\therem}{}
\newtheorem{term}[thm]{Term}
\renewcommand{\theterm}{}
\newtheorem{eg}[thm]{e.g.}
\newcounter{countntt}
\setcounter{countntt}{0}
\newtheorem{ntt}[countntt]{Notation}
\newcounter{countq}
\setcounter{countq}{0}
\newtheorem{q}[countq]{Question}
\newtheorem{prob}[thm]{Problem}
\newcounter{countproblem}
\setcounter{countproblem}{0}
\newtheorem{problem}[countproblem]{Problem}
% \setcounter{prob}{0}
\newtheorem{ans}[countq]{Answer}
\newtheorem{hint}{Hint}
\newtheorem{step}{Step}
\newtheorem{fact}[thm]{Fact}
\newtheorem{cnjc}{Conjecture}
\newtheorem{conv}{約束} %convention
\newtheorem{apdx}{余談} %appendix
\newtheorem{obs}{Observaton}
\newtheorem{note}{Note}
\newtheorem{memo}{Memo}
\newtheorem{claim}[thm]{Claim}
\newtheorem{cau}{Caution!}
\newtheorem{feeling}{気持ち}
\newtheorem{defprop}[thm]{Def-Prop}
\newtheorem{defthm}[thm]{Def-Thm}
\renewcommand{\thentt}{}
\renewcommand\proofname{\textit{proof}}
\renewcommand\qedsymbol{$\blacksquare$} %blacken halmos mark
% ProoF With Name emviromment
\newenvironment{pfwn}[1]{\noindent \textit{proof of #1.}\par}{ \rightline{\qedsymbol}} %you need to insert a blank line just before \end{pfwn} in oder to display qedsymbol
\newenvironment{skpf}{\noindent \textit{sketch of proof.}\par}{ \rightline{\qedsymbol}}
% no numbling--
\newtheorem*{defi*}{Def}
\newtheorem*{definition*}{Definition}
\newtheorem*{rem*}{Rem}
\newtheorem*{lem*}{Lemma}
\newtheorem*{cor*}{Cor}
\newtheorem*{thm*}{Thm}
\newtheorem*{theorem*}{Theorem}
\newtheorem*{teiri*}{定理}
\newtheorem*{prop*}{Prop}
\newtheorem*{tec*}{Technique}
\newtheorem*{term*}{Term}
\newtheorem*{ex*}{e.g}
\newtheorem*{rei*}{例}
\newtheorem*{prob*}{Problem}
\newtheorem*{q*}{Question}
\newtheorem*{ans*}{Answer}
\newtheorem*{hint*}{Hint}
\newtheorem*{fact*}{Fact}
\newtheorem*{apdx*}{余談} %appendix
\newtheorem*{cnjc*}{Conjecture}
\newtheorem*{conv*}{約束}
\newtheorem*{obs*}{Observaton}
\newtheorem*{note*}{Note}
\newtheorem*{cau*}{Caution!}
\newtheorem*{memo*}{Memo}
\newtheorem*{claim*}{Claim}
\newtheorem*{feeling*}{気持ち}
\newtheorem*{defprop*}{Def-Prop}
\newtheorem*{defthm*}{Def-Thm}
% ---
% Oprerators
\newcommand{\relmiddle}[1]{\mathrel{}\middle#1\mathrel{}}
% used as " \relmiddle| " streching \mid
\newcommand{\sforall}{{}^{\forall}}
\newcommand{\bcs}{\ \because\ }
\newcommand{\trf}{\ \therefore\ }
\newcommand{\otw}{\text{otherwise}}
\newcommand{\cst}{\text{const.}}
\newcommand{\whr}{\text{\ where\ }}
\newcommand{\defeq}{\coloneqq}
\newcommand{\inv}[1]{#1^{-1}}
\newcommand{\trans}[1]{{}^t #1}
\newcommand{\per}[1]{#1^{\perp}}
\newcommand{\lyama}{\langle}
\newcommand{\ryama}{\rangle}
\renewcommand{\bar}{\overline}
\newcommand{\cldots}{,\ldots ,}
\DeclareMathOperator{\Inn}{Inn}
\DeclareMathOperator{\Aut}{Aut}
\DeclareMathOperator{\End}{End}
\DeclareMathOperator{\Res}{Res}
\newcommand{\infint}{\int^{\infty}_{{-\infty}}}
\newcommand{\dint}{\displaystyle{\int}}
\renewcommand{\epsilon}{\varepsilon}
\renewcommand{\cal}[1]{\mathcal{#1}}
\renewcommand{\phi}{\varphi}
\newcommand{\lap}{\Delta}% need to fix
\DeclareMathOperator{\re}{Re}
\DeclareMathOperator{\im}{Im}
\DeclareMathOperator{\id}{id}
\DeclareMathOperator{\ev}{ev}
\DeclareMathOperator{\Ker}{Ker} %Physics package has this by \ker
\renewcommand{\ker}{\Ker}
\DeclareMathOperator{\codim}{codim}
\DeclareMathOperator{\rank}{rank}
\DeclareMathOperator{\diag}{diag}
\DeclareMathOperator{\pr}{pr}
\DeclareMathOperator{\Gal}{Gal}
\DeclareMathOperator{\homeo}{Homeo}
\DeclareMathOperator{\aut}{Aut}
\DeclareMathOperator{\ad}{ad}
\DeclareMathOperator{\Ad}{Ad}
\DeclareMathOperator{\map}{Map}
\DeclareMathOperator{\sign}{sign}
\DeclareMathOperator{\diam}{diam}
\DeclareMathOperator{\vol}{vol}
\DeclareMathOperator{\Vol}{Vol}
\DeclareMathOperator{\catset}{\mathbf{Set}}
\DeclareMathOperator{\nat}{\mathbf{N}}
\DeclareMathOperator{\zah}{\mathbf{Z}}
\DeclareMathOperator{\real}{\mathbf{R}}
\DeclareMathOperator{\cpx}{\mathbf{C}}
\DeclareMathOperator{\quat}{\mathbf{H}}
\DeclareMathOperator{\kor}{\mathbf{K}}
\DeclareMathOperator{\complex}{\mathbf{C}}
\DeclareMathOperator{\trs}{\mathbf{T}}
\DeclareMathOperator{\cD}{\mathcal{D}}
\newcommand{\prr}[1]{\mathbf{P}^{#1}\mathbf{R}}
\newcommand{\prc}[1]{\mathbf{P}^{#1}\mathbf{C}}
\newcommand{\nin}{\not\in}
\DeclareMathOperator{\opensub}{\underset{\text{open}}{\subset}}
\DeclareMathOperator{\closesub}{\underset{\text{closed}}{\subset}}
\DeclareMathOperator{\cptsub}{\underset{\text{cpt}}{\subset}}
\newcommand{\comp}[1]{{#1}^{\mathrm{c}}} %complement
\newcommand{\inte}[1]{{#1}^{\mathrm{o}}} %naibu interior
\DeclareMathOperator{\defarrow}{\overset{\text{def}}{\Longleftrightarrow}}
\DeclareMathOperator{\1to1arrow}{\overset{\text{1:1}}{\longleftrightarrow}}
\DeclareMathOperator{\LRarrow}{\Leftrightarrow}
\DeclareMathOperator{\leftaction}{\curvearrowright}
\DeclareMathOperator{\rightaction}{\curvearrowleft}
\DeclareMathOperator{\simarrow}{\overset{\sim}{\longrightarrow}}
\DeclareMathOperator{\homeoarrow}{\overset{\approx}{\longrightarrow}}
\newcommand{\ronarrow}[1]{\overset{#1}{\longrightarrow}}
\newcommand{\lonarrow}[1]{\overset{#1}{\longleftarrow}}
\newcommand{\inclusion}{\hookrightarrow}
\DeclareMathOperator{\alev}{\ \text{a.e.}\,}
\DeclareMathOperator{\st}{\ \text{s.t.}\, }
\DeclareMathOperator{\ie}{\ \text{i.e.}\,}
\newcommand{\cinf}{C^{\infty}\text{-級}\hspace{-0.3zh}}
\newcommand{\cinff}{C^{\infty}\hspace{-0.3zh}}
\newcommand{\cone}{C^{1}\text{-級}\hspace{-0.3zh}}
\newcommand{\conti}[1]{C^{#1}\text{-級}\hspace{-0.3zh}}
\newcommand{\contiwd}[2]{C^{#1}(#2)\hspace{-0.3zh}}
\newcommand{\ccontiwd}[2]{C_{0}^{#1}(#2)\hspace{-0.3zh}}
\newcommand{\llocwd}[2]{L_{\mathrm{loc}}^{#1}(#2)\hspace{-0.3zh}}
\newcommand{\Lp}{L^p}
\newcommand{\Lpwd}[2]{L^{#1} (#2)}
\newcommand{\Lpnorm}[2]{\|#2\|_{L^{#1}}}
\DeclareMathOperator{\loc}{loc}
\DeclareMathOperator{\supp}{supp}
\DeclareMathOperator{\diff}{Diff}
\DeclareMathOperator{\grad}{grad}
\renewcommand{\div}{\mathrm{div}}
\DeclareMathOperator{\rot}{rot}
\DeclareMathOperator{\dvol}{dvol}
\DeclareMathOperator{\dist}{dist}
\newcommand{\del}{\partial}
\newcommand{\dx}{\frac{\partial}{\partial x}}
\newcommand{\dd}{\Delta}
\renewcommand{\ge}{\mathfrak{g}}%\ge was used as \geq
\DeclareMathOperator{\ha}{\mathfrak{h}}
\DeclareMathOperator{\secx}{\mathfrak{X}}
\DeclareMathOperator{\secxl}{\mathfrak{X}_L} % left invariant vector field
\newcommand{\fundgrp}[1]{\pi_1{({#1})}}
\newcommand{\pfundgrp}[2]{\pi_1({#1},{#2})}
\DeclareMathOperator{\res}{Res}
\newcommand{\secfund}{I \hspace{-2.5pt} I}
\newcommand{\thrfund}{I \hspace{-2.5pt} I\hspace{-2.5pt} I}
% Matrices
\DeclareMathOperator{\tr}{tr}
\DeclareMathOperator{\Symm}{Symm}
\DeclareMathOperator{\GL}{GL}
\DeclareMathOperator{\Symmp}{\mathrm{Symm}_{\geq 0}}
\DeclareMathOperator{\M}{M}
\DeclareMathOperator{\U}{U}
\DeclareMathOperator{\SL}{SL}
\DeclareMathOperator{\Skew}{Skew}
\DeclareMathOperator{\Or}{O}
\DeclareMathOperator{\SO}{SO}
\DeclareMathOperator{\Uni}{U}
\DeclareMathOperator{\SU}{SU}
\DeclareMathOperator{\Psl}{PSL}
\DeclareMathOperator{\Sp}{Sp}
% \DeclareMathOperator{}{}
\DeclareMathOperator{\glie}{\mathfrak{gl}}
\DeclareMathOperator{\slie}{\mathfrak{sl}}
\DeclareMathOperator{\olie}{\mathfrak{o}}
\DeclareMathOperator{\sulie}{\mathfrak{su}}
% ---
\newcommand{\Kahler}{K{\"a}hler}
\newcommand{\Poincare}{Poincar{\'e}}
% ---

\usepackage{longtable}
\usepackage{nccmath}
% \usepackage[dvipdfmx]{graphicx}
\usepackage{mleftright}
\newcommand{\lbig}{\mleft} %parentheses
\newcommand{\rbig}{\mright}
\newcommand{\blvert}{\lbig\lvert}
\newcommand{\brvert}{\rbig\rvert}
\usepackage{bm}
% \usepackage{mmacells}
\usepackage[utf8]{inputenc}

\lstset{%
  language={C++},
  basicstyle={\ttfamily\small},%
  identifierstyle={\small},%
  commentstyle={\small\itshape},%
  keywordstyle={\small\bfseries},%
  ndkeywordstyle={\small},%
  stringstyle={\small\ttfamily},
  frame={tb},
  breaklines=true,
  columns=[l]{fullflexible},%
  numbers=left,%
  xrightmargin=0zw,%
  xleftmargin=3zw,%
  numberstyle={\scriptsize},%
  stepnumber=1,
  numbersep=1zw,%
  lineskip=-0.5ex%
}

% matirices in lines.it is better to use "pmatrix"
\makeatletter
\def\tpmatrix#1{
  \setbox\z@=\vtop{\normalbaselines\m@th
    \ialign{\hfil$##$\hfil&&\quad\hfil$##$\hfil\crcr
      \mathstrut\crcr
      \noalign{\kern-\baselineskip}
      #1\crcr
      \mathstrut\crcr
      \noalign{\kern-\baselineskip}
    }
  }
  \dimen\z@=\dp\z@
  \setbox\z@=\vbox to \ht\z@{
    \hbox{$\displaystyle \left(\,\vcenter{\unvbox\z@}\,\right)$}
    \vss
  }
  \dp\z@=\dimen\z@
  \box\z@
}
\catcode`\@=11
\newbox\matbox
\def\topmatrix#1{\setbox\matbox=\vtop{\normalbaselines\m@th % set the matrix in
    \ialign{\hfil$##$\hfil&&\quad\hfil$##$\hfil\crcr    % a \vtop so the
      \mathstrut\crcr\noalign{\kern-\baselineskip}      % first baseline
      #1\crcr\mathstrut\crcr\noalign{\kern-\baselineskip}}} % lines up.
  % get twice difference between baseline and centerline of inner matrix:
  \dimen255=\dp\matbox \advance\dimen255 by -\ht\matbox
  % Center matrix and surround with parentheses:
  \setbox\matbox=\hbox{$\left( \,\vcenter{\box\matbox}\,\right)$}
  % Correct for difference between baseline and centerline of parentheses:
  \advance\dimen255 by -\dp\matbox \advance\dimen255 by \ht\matbox
  % Lower centered matrix back to its proper baseline:
  \lower0.5\dimen255\box\matbox
}
\catcode`\@=12
\makeatother
% ---
% bibliography
\begin{comment}
  \renewenvironment{thebibliography}[1]
  {\section*{\huge\bibname}
    \list{\@biblabel{\@arabic\c@enumiv}}%
    {\settowidth\labelwidth{\@biblabel{#1}}%
      \leftmargin\labelwidth
      \advance\leftmargin\labelsep
      \@openbib@code
      \usecounter{enumiv}%
      \let\p@enumiv\@empty
      \renewcommand\theenumiv{\@arabic\c@enumiv}\small}%
    \sloppy
    \clubpenalty4000
    \@clubpenalty\clubpenalty
    \widowpenalty4000%
    \sfcode`\.\@m
    \lefthyphenmin=2\righthyphenmin=2%% ハイフネーションの条件を少し緩和
    \frenchspacing% 全体のspaceを均等に
  }
  {\def\@noitemerr
    {\@latex@warning{Empty `thebibliography' environment}}%
    \endlist}
  \let\@openbib@code\@empty
  \newcommand{\bibname}{参考文献}
\end{comment}
% --
\usepackage[dvipdfmx]{hyperref}% must be at the end of preemble file
\usepackage[dvipdfmx]{pxjahyper}
\hypersetup{
  colorlinks=true, % リンクに色をつけない設定
  bookmarks=true, % 以下ブックマークに関する設定
  bookmarksnumbered=true,
  pdfborder={0 0 0},
  bookmarkstype=toc
  anchorcolor=pink,        % アンカーテキストの色指定(デフォルトはblack)
  citecolor=[HTML]{460e44},           % 参考文献リンクの色指定(デフォルトはgreen)
  filecolor=magenta,       % ローカルファイルリンクの色指定(デフォルトはmagenta)
  linkcolor=[HTML]{008899},        % 作成しているpdfファイルのリンクの色(デフォルトはred)
  linkbordercolor={1 0 0}, % R G B リンクを囲むボックスの色(デフォルトは1 0 0)
  urlcolor=Fuchsia,         % 外部参照しているurlの色(デフォルトはmagenta)
}
% make citation bold
\makeatletter
\def\@cite#1#2{[\textbf{#1\if@tempswa , #2\fi}]}
\def\@biblabel#1{[\textbf{#1}]}
\makeatother


% for a map diagramicaly
\begin{comment}
  \begin{array}{ccc}
    M & \stackrel{\phi}{\longrightarrow} & M' \\
    \rotatebox{90}{$\in$} & & \rotatebox{90}{$\in$} \\
    p & \longmapsto & \phi(p)
  \end{array}
\end{comment}
% not good
% \usepackage{emath}
% \usepackage[dvipdfmx]{emathP}
% \usepackage{emathZ}
% collision macros vs emath: tr, conj