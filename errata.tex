\documentclass{jsarticle}

\title{$e^{\pi i}sode$ Vol.8 正誤表}
\date{2018年12月}
\author{東京大学理学部数学科有志}
\pagestyle{empty}

\begin{document}
\maketitle

% お詫び文
第69回駒場祭にて配布いたしました$e^{\pi i}sode$ Vol.8に下記の誤りがございました.
お詫びして訂正いたします.申し訳ございません.
また,ここに示したもの以外にも,数学的な誤りや誤植が存在する可能性があります.
予めご了承くださいませ.

% 正誤表
\begin{table}[h]
\begin{center}
\begin{tabular}{lll}
訂正箇所&誤&正 \\\hline
8ページ下から8行目&辺の中心を結んだら&面の中心を結んだら \\
14ページ下から2行目&これで,$g(z)$が&これで,$h(z),\;g(z)$が \\
14ページ脚注16の2行目&$e^w=sum^\infty_{n=0}\frac{w^n}{n!}$&$e^w=\sum^\infty_{n=0}\frac{w^n}{n!}$ \\
14ページ脚注17&この右辺を$h(z)$の...&無限積$\prod_{n=1}^\infty p_n$に対し,... \\
14ページ脚注18&無限積$\prod_{n=1}^\infty p_n$に対し,...&右辺をそれぞれ$h(z),\;g(z)$の... \\
15ページ下から2行目&$0\leq y\geq 1$&$0\leq y\leq 1$ \\
21ページにある素数定理の主張&$\displaystyle\lim_{x\to\infty}\frac{x\cdot\pi(x)}{\log x}=1$&$\displaystyle\lim_{x\to\infty}\frac{\log x\cdot\pi(x)}{x}=1$ \\
22ページにあるリーマン予想の主張&$\zeta{s}=0$&$\zeta(s)=0$
\end{tabular}
\end{center}
\end{table}

\end{document}