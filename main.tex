\documentclass[notombow,episode,openany,dvipdfmx]{kyouritu}
\usepackage{graphicx,color}
\usepackage{tikz}
\usepackage{makeidx,multicol}
\usepackage{amsmath,amssymb,amsthm}
\usepackage{standalone}
\usepackage{tocloft}
\usepackage{aliascnt}
\usepackage{ascmac}
\usepackage{subfiles}


%inoue.texより移動
\usepackage{wrapfig}
\usepackage[all]{xy}




%Section等先頭を大文字にすると番号付けしない.
\newcommand{\Chapter}[1]{\chapter*{{\Huge #1}}
\markboth{#1}{#1}
\addcontentsline{toc}{chapter}{#1}
\stepcounter{chapter}}
\newcommand{\Section}[1]{\section*{{\huge #1}}
\addcontentsline{toc}{section}{#1}}
\newcommand{\Subsection}[1]{\subsection*{\underline{#1}}}
\newcommand{\Subsubsection}[1]{\subsubsection*{#1}}
\setcounter{tocdepth}{0} %Chapterのみ表示する
\renewcommand*{\sectionmark}[1]{}
\renewcommand{\thesection}{\arabic{section}}
\renewcommand{\thesubsection}{\thesection.\arabic{subsection}}
\renewcommand{\theequation}{\arabic{equation}}
\renewcommand*{\KBsectionfont}[1]{\normalfont\bfs\huge #1}



%本文
\begin{document}
\frontmatter
\Chapter{まえがき}

本日は「ますらぼ」にお越しいただき,誠にありがとうございます.
本企画は数学科の有志により企画・運営されているもので,今年で6年目となります.\\
 この冊子「$e^{\pi i}sode$」も有志によって制作・頒布されているもので,今回でvol.8です.歴代の$e^{\pi i}sode$\footnote{名作ぞろいである.本当に.}見本も会場にございますので,よろしければそちらもご覧ください.\\
 今回は「確率論への招待」「正多面体」「リーマンゼータ関数について」と,どれも非常に身近な話題となっています.一つ一つその面白さをこのまえがきで紹介したいのですが,まえがきだけ読んで満足\footnote{理学書あるあるである.}していただいてはせっかくの本文がもったいないので,あえて何も触れないことにします.今回も五月祭に引き続き,パズルのコーナーをご用意いたしました.\\
 大変短いあいさつで恐縮でございますが,数学科のホームグラウンド\footnote{数学科の建物「数理科学研究科棟」は本郷ではなくここ駒場にある.}であるこの駒場で,駒場祭を,ますらぼを,そして$e^{\pi i}sode$を,お楽しみください\footnote{また,「ますらぼ」と直接は関係ありませんが11/23(金)の午後には,数理科学研究科棟にて,一般向けの公開講座「行列」がございますので,よろしければお立ち寄りください.}!(執印)

\clearpage % tocloft パッケージを使う場合は自分で \clearpage しないといけない
\tableofcontents
\cleardoublepage % 本文を見開き右から始めるために追加(ヘッダのページ数表示の関係上右始めの必要あり) 2018/7/10 by Okudenn
\mainmatter
\subfile{manuscript/hamada}
\subfile{manuscript/inoue}
\subfile{manuscript/imai}
\subfile{manuscript/morikawa}
% \cftaddtitleline{toc}{chapter}{圏論が分かる4コマ漫画(小林)}{章間}
% このようにすると好きなように目次を変更できる
\backmatter
%\Chapter{編集後記}を入れても良い
\thispagestyle{empty}
\vspace*{10zw}
\vfill

\parindent=0pt
\begin{picture}(110,1)
\setlength{\unitlength}{1truemm}
\put(5,2){\Large\textbf{$e^{\pi i}sode$ Vol.8 }} 
\thicklines
\put(0,1){\line(2,0){110}}
\thinlines
\put(0,0){\line(2,0){110}}
\end{picture}

\small{2018年11月23日発行}\\
 \normalsize{著 者・・・・・東京大学理学部数学科有志}\\
 \normalsize{発行人・・・・・執印 剛史}\\
\begin{picture}(100,1)
\setlength{\unitlength}{1truemm}
\thinlines
\put(0,1){\line(2,0){110}}
\thicklines
\put(0,0){\line(2,0){110}}
\put(0,-5){\small{\copyright  Students at Department of Mathematics,The University of Tokyo 2018 Printed in Japan}}
\end{picture}

\end{document}
