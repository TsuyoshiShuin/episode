\Chapter{まえがき}

本日は「ますらぼ」にお越しいただき,誠にありがとうございます.
本企画は数学科の有志により企画・運営されているもので,今年で6年目となります.\\
 この冊子「$e^{\pi i}sode$」も有志によって制作・頒布されているもので,今回でvol.8です.歴代の$e^{\pi i}sode$\footnote{名作ぞろいである.本当に.}見本も会場にございますので,よろしければそちらもご覧ください.\\
 今回は「確率論への招待」「正多面体」「リーマンゼータ関数について」と,どれも非常に身近な話題となっています.一つ一つその面白さをこのまえがきで紹介したいのですが,まえがきだけ読んで満足\footnote{理学書あるあるである.}していただいてはせっかくの本文がもったいないので,あえて何も触れないことにします.今回も五月祭に引き続き,パズルのコーナーをご用意いたしました.\\
 大変短いあいさつで恐縮でございますが,数学科のホームグラウンド\footnote{数学科の建物「数理科学研究科棟」は本郷ではなくここ駒場にある.}であるこの駒場で,駒場祭を,ますらぼを,そして$e^{\pi i}sode$を,お楽しみください\footnote{また,「ますらぼ」と直接は関係ありませんが11/23(金)の午後には,数理科学研究科棟にて,一般向けの公開講座「行列」がございますので,よろしければお立ち寄りください.}!(執印)
